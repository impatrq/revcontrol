\documentclass[a4paper,12pt]{report}
\usepackage{graphicx}
\usepackage{hyperref}
\usepackage[spanish]{babel}
\usepackage{fancyhdr}
\usepackage{tocloft}
\usepackage{setspace}
\usepackage{ragged2e}
% Permite el uso de títulos, encabezados y pies de página.
\usepackage{titlesec}
% Permite la inserción de imágenes al archivo.
\usepackage{graphicx}
% Permite el uso de imágenes y figuras rotadas
\usepackage{rotating}
% Permite cambiar el tamaño de las imágenes.
\usepackage[export]{adjustbox}
% Permite el uso de hojas apaisadas y verticales.
\usepackage{pdflscape}
% Permite el uso de múltiples imágenes dentro de una figura.
\usepackage{subcaption}
% Permite la definición de objetos flotantes como tablas e imágenes.
\usepackage{float}
% Expande las capacidades de las tablas.
\usepackage{array}
% Expande las capacidades de las tablas.
\usepackage{tabularray}
\usepackage{appendix}
\usepackage{tabularx}% Permite el uso de ítems con diferente forma.
\usepackage{enumitem}
\usepackage{xcolor} % Asegúrate de incluir este paquete
\definecolor{Celeste}{HTML}{00B5BE} % Reemplaza '00B5BE' por el código que desees

% Configuración de márgenes y encabezados
\usepackage[a4paper, margin=2.5cm]{geometry}
\pagestyle{fancy}
\fancyhf{}
\fancyhead[L]{Rev-Control}
\fancyhead[R]{\leftmark}
\fancyfoot[C]{\thepage}

% Establece que en la tabla de contenidos se use una línea de puntos para marcar el número de página.
\renewcommand{\cftsecleader}{\cftdotfill{\cftdotsep}}

\begin{document}

% Carátula
\begin{titlepage}
    \centering
    % Imagen rectangular grande y estrecha centrada
    \begin{figure}
        \centering
        \includegraphics[width=1.1\textwidth, height=2.3cm]{Imagenes/logos2.png} % Reemplaza por tu imagen
    \end{figure}
    
    % Logo del proyecto
    \includegraphics[width=0.7\textwidth]{Imagenes/LOGO REV CONTROL OFICIAL.png} 
    \vspace{1cm}
    
    {\Huge \textbf{\textcolor{Celeste}{Carpeta Técnica}\\}}
    \vspace{1cm}
    {\Large Curso: 7° 1° Aviónica\\}
    \vspace{0.5cm}
    {\Large Comisión: C}
    \vspace{1cm}

    \textbf{Pagina web:} \href{https://www.google.com/}{Link de acceso}\\
    \textbf{Trello:} \href{https://trello.com/b/yDSPDlAp/kanban}{Link de acceso}\\
    \textbf{Github:} \href{https://github.com/impatrq/revcontrol}{Link de acceso}\\
    \textbf{Redes Sociales:} \href{https://www.instagram.com/rev.control/}{Link de acceso} \\
    \vfill
    \textbf{IMPA TRQ E.E.S.T N°7 2024}
\end{titlepage}

%\renewcommand{\contentsname}{Índice general}
\tableofcontents
\newpage

% 2. Presentación del equipo
\chapter{Preámbulo}
    \section{Integrantes}
    
        \begin{itemize}
            \item Gonzalo Acosta
            \item Lautaro Alfaro
            \item Leonardo He
            \item Tadeo Ibaceta
            \item Marcos Martinez
            \item Juan Quintero
            \item Santiago Flores
        \end{itemize}
                    
    
    \section{Foto de cada integrante}
    Información\par
    
    \section{Foto grupal}
    Información\\
    
    \section{Horas dedicadas por cada integrante (registro personal)}
    Información\\
% 4. Introducción
\chapter{Introducción}

\section{Objetivo}  
El objetivo de \textbf{REV-CONTROL} es ofrecer una solución innovadora y accesible para técnicos y mecánicos, permitiendo medir de manera eficiente y precisa los parámetros críticos de un motor. Este banco de prueba portátil, REV-CONTROL, facilita el diagnóstico y mantenimiento, utilizando tecnología de punta de amplificación y filtración de señales para proporcionar datos en tiempo real sobre las condiciones del motor.

\section{Descripción de la solución buscada}  
REV-CONTROL es un banco de prueba portátil diseñado para obtener datos precisos sobre los parámetros esenciales de cualquier motor. Con su diseño compacto y fácil de usar, REV-CONTROL está optimizado para realizar mediciones de revoluciones por minuto (RPM), presión y temperatura del aceite, temperatura de la cabeza de cilindro, temperatura de agua y concentración de oxígeno en los gases. Su sistema permite a los técnicos tomar decisiones rápidas y acertadas para garantizar el correcto funcionamiento de los motores.

\section{Segmento destino y alcance}  
El público objetivo de \textbf{REV-CONTROL} son los técnicos y mecánicos que trabajan en mantenimiento y diagnóstico de motores, desde vehículos de carretera hasta equipos industriales. Con un alcance global, este sistema busca transformar el modo en que se gestionan las pruebas de motor, especialmente en situaciones donde la portabilidad, la rapidez y la precisión son esenciales.

\section{Captura representativa del proyecto}

\begin{figure}[H]
    \centering
    \includegraphics[width=0.6\textwidth]{Imagenes/Monitor y maletin-min.png}
    \caption{Imagen representativa del proyecto REV-CONTROL en funcionamiento.}
    \label{fig:representativa}
\end{figure}

\section{Diagrama en bloques del proyecto}

\begin{figure}[H]
    \centering
    \begin{tikzpicture}[
      block/.style={rectangle, draw, text centered, minimum height=1cm, minimum width=3cm, rounded corners},
      arrow/.style={-{Latex[length=3mm, width=2mm]}, thick},
      node distance=1.5cm
    ]
      
      % Definición de nodos
      \node[block, fill=cyan!30] (motor) {Motor en funcionamiento};
      \node[block, fill=green!30, below=of motor] (sensores) {Sensores de parámetros críticos};
      \node[block, fill=blue!30, below=of sensores] (condicionamiento) {Condicionamiento de señales};
      \node[block, fill=cyan!30, below=of condicionamiento] (lpc845) {Microcontrolador LPC845};
      \node[block, fill=green!30, below=of lpc845] (alarma) {Sistema de alarma};
      \node[block, fill=green!30, right=of lpc845] (monitor) {Monitor Kinseal};

      % Conexiones con flechas
      \draw[arrow] (motor) -- (sensores);
      \draw[arrow] (sensores) -- (condicionamiento);
      \draw[arrow] (condicionamiento) -- (lpc845);
      \draw[arrow] (lpc845) -- (alarma);
      \draw[arrow] (lpc845) -- (monitor);

    \end{tikzpicture}
    \caption{Diagrama en bloques del sistema REV-CONTROL.}
    \label{fig:diagrama_bloques}
\end{figure}

\section{Resultado conseguido}  
El resultado de \textbf{REV-CONTROL} es una herramienta integral para diagnóstico de motores, que ofrece mediciones precisas y en tiempo real de parámetros críticos. Su sistema permite a los técnicos obtener datos detallados de manera clara y sencilla, mejorando la eficacia en las reparaciones y mantenimiento de motores, además de ofrecer un sistema de alarmas para alertar de situaciones críticas en los parámetros monitoreados.



% 5. Software
\chapter{Software}

\section{Códigos significativos}
A continuación se adjunta el código de lectura de pines en \hyperref[adc_code]{ADC} y \hyperref[spi_code]{SPI}:

% Fragmento de código SPI
\lstinputlisting[
    language=C,
    caption=Software de RevControl, 
    label=spi.code
]{rev-control_spi.c}

\newpage

\section{Interfaz visual}

Esta interfaz está diseñada para monitorear en tiempo real los parámetros críticos del motor, específicamente el \textbf{Rotax 912 ULS}. A continuación, se describe cada una de las secciones principales:

\begin{itemize}
    \item \textbf{Indicador de RPM (centro):}
    \begin{itemize}
        \item Medidor circular que muestra las revoluciones por minuto del motor.
        \item Escala dividida en zonas de colores:
        \begin{itemize}
            \item \textbf{Verde:} Rango seguro de operación.
            \item \textbf{Amarillo:} Precaución.
            \item \textbf{Rojo:} Zona de riesgo, indicando revoluciones excesivas.
        \end{itemize}
    \end{itemize}

    \item \textbf{Indicadores de temperatura de las cabezas de cilindros (izquierda):}
    \begin{itemize}
        \item Cuatro barras verticales que representan la temperatura de las cabezas de los \textbf{cuatro cilindros} del motor.
        \item Los colores degradan de \textbf{verde} a \textbf{rojo}, indicando el nivel de temperatura en relación al rango seguro.
    \end{itemize}

    \item \textbf{Concentración de oxígeno (abajo izquierda):}
    \begin{itemize}
        \item Muestra un valor que indica la proporción de oxígeno en la mezcla de gases.
        \item Relacionado con la eficiencia de la combustión interna del motor.
    \end{itemize}

    \item \textbf{Temperatura del agua (centro abajo):}
    \begin{itemize}
        \item Indicador dedicado a la temperatura del refrigerante líquido del motor.
        \item Utiliza un rango de colores para reflejar el estado del sistema de enfriamiento.
    \end{itemize}

    \item \textbf{Indicador de temperatura del aceite (derecha abajo):}
    \begin{itemize}
        \item Medidor circular que muestra la temperatura actual del aceite lubricante.
        \item Escala que incluye zonas de operación segura y rangos críticos por sobrecalentamiento.
    \end{itemize}

    \item \textbf{Presión de aceite (derecha):}
    \begin{itemize}
        \item Sección dedicada a verificar si la presión del aceite está en el rango adecuado.
        \item Indicadores gráficos para alertar al usuario:
        \begin{itemize}
            \item \textbf{Check verde:} Presión adecuada.
            \item \textbf{Cruz roja:} Falla o presión insuficiente.
        \end{itemize}
    \end{itemize}

    \item \textbf{Descripción del motor (arriba derecha):}
    \begin{itemize}
        \item Cuadro informativo con detalles técnicos del \textbf{Rotax 912 ULS}.
        \item Destaca su potencia, sistema de refrigeración mixta (agua y aire) y eficiencia en el consumo de combustible.
    \end{itemize}

    \item \textbf{Reloj (parte superior):}
    \begin{itemize}
        \item Muestra la hora actual, útil para registro de eventos o sincronización con otras operaciones.
    \end{itemize}
\end{itemize}

    
\section{Estructuras de datos}

Esta sección describe las principales estructuras de datos utilizadas para gestionar las lecturas de los sensores y la comunicación SPI en el sistema. Se incluye una descripción de las variables globales, las funciones clave, y las estructuras de configuración para el ADC y el SPI.

\subsection{Variables globales}

Las variables globales se emplean para almacenar la configuración y los resultados de las conversiones ADC y SPI. A continuación, se describen las más importantes:

\begin{itemize}
    \item \textbf{adc\_channel[3]}: Un \textit{array} de 8 bits que define los canales ADC en uso. Cada elemento representa un canal específico:
        \begin{itemize}
            \item \texttt{ADC0\_CH1}: Canal para la concentración de oxígeno.
            \item \texttt{ADC0\_CH2}: Canal para las RPM.
            \item \texttt{ADC0\_CH3}: Canal para la presión de aceite.
        \end{itemize}
        
    \item \textbf{channel\_result[3]}: Un \textit{array} de enteros de 16 bits que almacena los resultados de la conversión ADC para cada canal.

    \item \textbf{lambda}, \textbf{oil\_pressure}, \textbf{RPM}: Variables de 16 bits donde se guardan los valores procesados de cada lectura del canal, correspondientes a la concentración de oxígeno, la presión de aceite y las RPM, respectivamente.

    \item \textbf{CS[7]}: Un \textit{array} de enteros de 8 bits que representa los pines de Chip Select (CS) utilizados para manejar distintos dispositivos SPI conectados al sistema.
\end{itemize}

\subsection{Funciones y prototipos principales}

Estas funciones organizan el flujo del programa, facilitando la adquisición y transmisión de datos desde los sensores al sistema:

\begin{itemize}
    \item \textbf{void ADC\_Configuration(void)}: Configura el ADC, activando los canales necesarios y definiendo los parámetros de conversión.

    \item \textbf{void spi\_cs\_low(void)} y \textbf{void spi\_cs\_high(void)}: Controlan el estado de los pines de Chip Select (CS) para habilitar o deshabilitar el dispositivo SPI según sea necesario.

    \item \textbf{float max6675\_get\_temp(void)}: Realiza una lectura de temperatura desde un sensor MAX6675 a través de SPI, devuelve el valor en grados Celsius.
\end{itemize}

\subsection{Estructura de configuración para el SPI}

La estructura \texttt{spi\_transfer\_t} se utiliza para definir los parámetros de la transferencia SPI. Esto organiza la lectura del sensor de temperatura y otros dispositivos conectados:

\begin{itemize}
    \item \texttt{txData}: Puntero a los datos que se van a transmitir (NULL si no se envían datos).
    \item \texttt{rxData}: Puntero a los datos recibidos en la transferencia.
    \item \texttt{dataSize}: Tamaño de la transferencia en bytes (en este caso, 2 bytes).
    \item \texttt{configFlags}: Indicadores de configuración que controlan la finalización de la transferencia y el formato de los datos.
\end{itemize}

\subsection{Estructura de configuración para el ADC}

La estructura \texttt{adc\_conv\_seq\_config\_t} define el modo de conversión y los canales activos para el ADC. Los campos principales son:

\begin{itemize}
    \item \texttt{channelMask}: Máscara de bits que activa los canales especificados en el ADC.
    \item \texttt{triggerMask}: Define el disparo de conversión, ya sea en modo automático o por software.
    \item \texttt{interruptMode}: Configura las interrupciones que se activan cuando la conversión en un canal finaliza.
\end{itemize}

Estas estructuras de datos, variables y funciones permiten un manejo organizado y eficiente de las lecturas y configuraciones de los sensores en el sistema embebido.

 

    

% 6. Sistema embebido
\chapter{Sistema embebido}
\section{Microcontroladores y/o microprocesadores utilizados}
Información

\section{Placas de desarrollo}
Información

\section{Software utilizados para el desarrollo}
Información

\section{Diagrama en bloque de la solución}
Información

\section{Lenguajes de programación usados}
Información

\section{Capturas de códigos significativos}
Información

\section{Especificación de periféricos utilizados}
Información

\section{Estructuras de datos}
Información

% 7. Electrónica
\chapter{Electrónica}
\section{Diagrama en bloque de las partes}
Información

\section{Software usado para el desarrollo de esquemáticos y PCB}
Información

\section{Esquemático de cada bloque}
Información

\section{PCB de cada bloque}
Información

\section{Modelo 3D de cada PCB}
Información

\section{Especificaciones sobre fuentes de alimentación y potencias}
Información

\section{Especificaciones técnicas de los componentes}
Información

% 8. Estructura
\chapter{Estructura}
\section{Diagrama general de la estructura}
Información

\section{Software de diseño utilizado}
Información

\section{Descripción de cada parte de la estructura}
Información

\section{Imágenes exportadas de los diseños}
Información

% 9. Anexo
\begin{appendix}
   \chapter{Apéndice A: Esquemáticos}
    Este apéndice presenta los esquemáticos eléctricos desarrollados para el proyecto . Los esquemáticos incluidos proporcionan una visión detallada de los circuitos clave que permiten el funcionamiento eficiente del sistema. Estos esquemáticos son fundamentales para comprender la implementación de los sistemas de control y gestión de energía, y sirven como guía técnica para la construcción y funcionamiento del prototipo.
    Los dos esquemáticos que se incluyen en este apéndice son:\par
    
    \begin{itemize} [label = -]
    \setlength{\itemindent}{2em}
        \item MPPT (Maximum Power Point Tracking): Este esquema muestra el diseño del sistema encargado de optimizar la recolección de energía desde los paneles solares, asegurando que el sistema opere en el punto de máxima eficiencia, maximizando la energía capturada y transferida a la batería.\par
        \item Sistema de Control: Este esquema describe el hardware y la lógica de control del sistema de batería gravitatoria, incluyendo los componentes encargados de gestionar el flujo de energía entre la batería y la carga, así como los sensores que permiten un control preciso del sistema.\par
    \end{itemize}
    
    Ambos esquemáticos son esenciales para el correcto funcionamiento del sistema y proporcionan las bases técnicas para futuras expansiones o modificaciones en el diseño.\par

    \begin{landscape}
        \begin{sidewaysfigure}
            \centering
            \includegraphics[angle=270, width=\textwidth,height=\textheight,keepaspectratio]{Anexo-A Bloque/Rev-Control Version 2.5 Parte 1.PDF}
            \label{fig:A_1}
        \end{sidewaysfigure}
        
        \newpage
        
        \begin{sidewaysfigure}
            \centering
            \includegraphics[angle=270, width=\textwidth,height=\textheight,keepaspectratio]{Anexo-A Bloque/Rev- Control Version 2.4 Parte 2.pdf}
            \label{fig:A_2}
        \end{sidewaysfigure}
        
    \end{landscape}
   \chapter{Apéndice B: Documentación-ROTAX}
    La información presentada aquí es el certificado de Aeronavegabilidad y documentación técnica del motor trabajado en este proyecto (Motor ROTAX 912 ULS). Para presentar la bases de información que utilizamos para este proyecto respecto a los parámetros. A continuación se va adjuntar las fuentes de adquisición de dichos documentos y fragmentos de estos mismos:
    
\begin{itemize}
    \item Certificado de Aeronavegabilidad: 
    \href{https://www.seguridadaerea.gob.es/sites/default/files/HD%20TC286-I%20r8.pdf}{www.seguridadaerea.gob.es} \\
        \hyperlink{certificado-aeronavegabilidad}{-Salto a pagina-} % Enlace directo a la primera página del certificado
    
    \item Documentación Técnica 
        \href{https://www.flyrotax.com/p/service/technical-documentation}{www.flyrotax.com} \\

    \begin{itemize}
        \item SERVICE INSTRUCTION - PAC (Pag. 8) 
            \hyperlink{service-instruction-pag8}{-Salto a pagina-} % Enlace directo a la página 8 en el documento
    \end{itemize}

    \begin{itemize}
        \item MAINTENANCE MANUAL LINE (Pag. x)
    \end{itemize}

    \begin{itemize}
        \item OPERATORS MANUAL LINE (Pag. x)
    \end{itemize}

    \begin{itemize}
        \item MAINTENANCE MANUAL (Pag. x)
    \end{itemize}

    \begin{itemize}
        \item USER MANUAL (Pag. x)
    \end{itemize}
\end{itemize}

% Insertar el PDF completo a continuación de la descripción
\begin{landscape}
    % Insertar el PDF del certificado y establecer un marcador en la primera página
    \includepdf[pages=1-9, scale=0.9, pagecommand={\hypertarget{certificado-aeronavegabilidad}{}}]{Anexo-B Bloque/HD TC286-I r8.pdf}
    % Este es el Certificado de Aeronavegabilidad con el marcador para el enlace
\end{landscape}

\begin{landscape}
    % Insertar el PDF del Service Instruction con marcador en la página 6
    \includepdf[pages=6, scale=0.9, pagecommand={\hypertarget{service-instruction-pag8}{}}]{Anexo-B Bloque/SI-PAC-027.pdf}
    % Este es el Service Instruction en la página 6 con el marcador para el enlace
\end{landscape}
\end{appendix}



\end{document}